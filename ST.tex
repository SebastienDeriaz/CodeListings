\usepackage{listings}

\usepackage[dvipsnames]{xcolor}

\lstdefinestyle{ST}{
	keywords=[1]{VAR, CONSTANT, END_VAR, PROGRAM, END_PROGRAM, VAR_GLOBAL, RETAIN, VAR_INPUT, VAR_OUTPUT},
	keywords=[2]{AT, IF, THEN, ELSIF, END_IF, ELSE, UINT, BOOl, INT, DREAL, REAL, DINT, TIME, SINT, LINT, USINT, UDINT, ULINT, TIME_OF_DAY, DATE_AND_TIME, DATE, STRING, LREAL, CASE},
	keywords=[3]{:=,>,<,=,<=,>=,;,OR,AND},
	keywordstyle=[1]\bfseries\color{blue},
	keywordstyle=[2]\bfseries\color{RoyalBlue},
	keywordstyle=[3]\bfseries\color{RoyalBlue},
	backgroundcolor=\color{white},   % choose the background color; you must add \usepackage{color} or \usepackage{xcolor}; should come as last argument
	basicstyle=\footnotesize\ttfamily\color{black},        % the size of the fonts that are used for the code
	belowcaptionskip=1\baselineskip,
	breakatwhitespace=false,         % sets if automatic breaks should only happen at whitespace
	breaklines=true,                 % sets automatic line breaking
	captionpos=b,                    % sets the caption-position to bottom
	commentstyle=\color{green!40!black},    % comment style
	deletekeywords={...},            % if you want to delete keywords from the given language
	escapeinside={\%*}{*)},          % if you want to add LaTeX within your code
	extendedchars=true,              % lets you use non-ASCII characters; for 8-bits encodings only, does not work with UTF-8
	firstnumber=1,                % start line enumeration with line 1000
	frame=leftline,	                   
	%identifierstyle=\color{blue!50!black},
	keepspaces=true,                 % keeps spaces in text, useful for keeping indentation of code (possibly needs columns=flexible)
	%language=C,                 % the language of the code
	%morekeywords={*,...},            % if you want to add more keywords to the set
	numbers=left,                    % where to put the line-numbers; possible values are (none, left, right)
	numbersep=10pt,                   % how far the line-numbers are from the code
	numberstyle=\tiny\color{gray}, % the style that is used for the line-numbers
	rulecolor=\color{black},         % if not set, the frame-color may be changed on line-breaks within not-black text (e.g. comments (green here))
	morecomment=[n]{(*}{*)},
	morecomment=[n][\color{magenta}]{\%I}{\ },
	morecomment=[n][\color{magenta}]{\%Q}{\ },
	showspaces=false,                % show spaces everywhere adding particular underscores; it overrides 'showstringspaces'
	showstringspaces=false,          % underline spaces within strings only
	showtabs=false,                  % show tabs within strings adding particular underscores
	stepnumber=1,                    % the step between two line-numbers. If it's 1, each line will be numbered
	stringstyle=\color{orange},
	tabsize=2,	                   % sets default tabsize to 2 spaces
	title=\lstname,
	xleftmargin=1.5cm,
}